\documentclass[a4paper,10pt,twoside,titlepage,onecolumn]{book}
\pdfcompresslevel 9

\usepackage{ucs}
\usepackage[utf8x]{inputenc}
\usepackage[francais]{babel}
\usepackage[T1]{fontenc}
\usepackage[pdftex]{graphicx}

\usepackage[pdftex]{hyperref}
\hypersetup{%
   unicode=true,
   colorlinks=true, 
   linkcolor=black,
   urlcolor=blue,
   pdftitle={Manuel de l'utilisateur},
   pdfsubject={Manuel d'utilisation du logiciel XINX, editeur de feuille de style},
   pdfauthor={Ulrich Van Den Hekke}
}

\author{Ulrich Van Den Hekke}
\date{05/05/2011}
\title{Manuel de l'utilisateur}

\begin{document}
  
\maketitle

\tableofcontents \addcontentsline{toc}{chapter}{Table des matières}

\chapter{Introduction}

XINX est un logiciel de développement essentielement tournée vers l'édition de feuille de style XSL, utilisé pour générer des fichiers HTML. 
Une feuille de style XSL est un fichier qui décrit une liste de transformation à appliquer à un fichier XML pour le transformer en un autre fichier text (Text plat, HTML, ou un autre fichier XML).
\\

Le logiciel a été écris au début pour le développement des feuilles de styles de la société \href{http://www.generixgroup.com/}{Generix Group} et été fortement tournée vers cette technologie. Maintenant XINX se concentre sur le développement de feuille de style, et tout spécificité lié à la société Generix Group a été déporté dans un plugins.
\\

Le logiciel XINX peut voir ses fonctionnalité étendue à l'aide de modèle (parfois aussi appelé snipet ou template), à l'aide de script (au format ECMAScript, proche de qu'est la JavaScript), ou à l'aide de plugins (écrit en C++ et en utilisant le framework Qt).
\\

XINX utilise le framework \href{http://qt.nokia.com}{Qt} comme base. Ce même framework est utilisé pour développé l'environnement de bureau \href{htpp://ww.kde.org}{KDE}, mais aussi des logiciels connues comme Skype, \ldots. Qt est n'est pas seulement un framework graphique mais propose quelques extentions au langauge C++ à l'aide des signaux, des slots, des pointeurs partagées, des listes, des boucles foreach, \ldots.
\\

Ce manuel à pour but de vous aider à utiliser le logiciel XINX au mieux. 

\chapter{Installation de XINX}

\section{MS/Windows}

\subsection{Télécharger XINX}

\subsubsection{Version binaire}

Vous pouvez obtenir la version binaire de XINX pour les systèmes d'exploitations Ms/Windows.

\subsubsection{Version source}

Vous pouvez récupérer les sources de XINX sur la page de téléchargement à l'adresse \url{http://xinx.shadoware.org/wiki/Download}. Le fichier se présentera sous forme d'une archive \emph{.tar.gz} ou d'une archive \emph{.7z}. Vouz pourrez décompresser l'archive à l'aide de la commande :

\begin{verbatim}
7z x xinx-0.10.1.2581.7z
\end{verbatim}

Vous pouvez également télécharger les sources de XINX à l'aide de \href{http://subversion.tigris.org/}{SubVersion}\footnote{SubVersion est un gestionnaire de version, sous Windows il est possible d'utiliser TortoiseSVN. Sous Gnu/Linux, vous pouvez utiliser kdesvn par exemple}.

Pour récupérer les sources de la version de développement de XINX, il est possible d'utiliser la ligne de commande suivante :

\begin{verbatim} 
svn co http://svn.shadoware.org/xinx/trunk
\end{verbatim}

Les versions stables sont dans le dossiers tags et branches. La sortie de chaque nouvelle version engendre la création d'une branche (v suivis du numéro de version). Les corrections de bugs propre à la version sont effectué dans cette branche. La sortie d'une nouvelle release (mise à jours de la version), engendre la création d'un tag (r suivis du numéro de release). On considère qu'une version est écrite ainsi : v0.10 (avec 3 nombres qui se suivent), et une release est écrite ainsi : r0.10.1 (le troisième indiquant le numéro de la correction).

\section{Gnu/Linux}

\subsection{Télécharger XINX}

\subsubsection{Version binaire pour Gnu/Debian}

Une version binaire de XINX est disponible pour la distribution Gnu/Debian. Vous pouvez la télécharger en ajoutant le dépôt suivant à votre fichier \emph{/etc/apt/sources.list} :

\begin{verbatim}
deb http://apt.shadoware.org/ squeeze main
\end{verbatim}

et en executant la commande :

\begin{verbatim}
aptitude install xinx
\end{verbatim}

Vous pouvez alors démarrer XINX depuis le menu ou 

\subsubsection{Version source}

Vous pouvez récupérer les sources de XINX sur la page de téléchargement à l'adresse \url{http://xinx.shadoware.org/wiki/Download}. Le fichier se présentera sous forme d'une archive \emph{.tar.gz} ou d'une archive \emph{.7z}. Vouz pourrez décompresser l'archive à l'aide de la commande :

\begin{verbatim}
7z x xinx-0.10.1.2581.7z
\end{verbatim}

Vous pouvez également télécharger les sources de XINX à l'aide de \href{http://subversion.tigris.org/}{SubVersion}\footnote{SubVersion est un gestionnaire de version, sous Windows il est possible d'utiliser TortoiseSVN. Sous Gnu/Linux, vous pouvez utiliser kdesvn par exemple}.

Pour récupérer les sources de la version de développement de XINX, il est possible d'utiliser la ligne de commande suivante :

\begin{verbatim} 
svn co http://svn.shadoware.org/xinx/trunk
\end{verbatim}

Les versions stables sont dans le dossiers tags et branches. La sortie de chaque nouvelle version engendre la création d'une branche (v suivis du numéro de version). Les corrections de bugs propre à la version sont effectué dans cette branche. La sortie d'une nouvelle release (mise à jours de la version), engendre la création d'un tag (r suivis du numéro de release). On considère qu'une version est écrite ainsi : v0.10 (avec 3 nombres qui se suivent), et une release est écrite ainsi : r0.10.1 (le troisième indiquant le numéro de la correction).


\subsubsection{Compilation de XINX à partir des sources}

Sous Gnu/Linux la compilation necessite l'installation des paquets suivante\footnote{les paquets sont ceux définit dans Gnu/Debian Squeeze, si votre distribution est différente, il vous faudra adapter la selection des paquets} :
\begin{itemize}
 \item libxml2-dev
 \item libxslt1-dev
 \item cmake
 \item libqt4-dev
 \item libsvncpp-dev 
\end{itemize}


\chapter{Interface principale de XINX}

\section{Configuration de XINX}

\section{Les projets}

\subsection{Création d'un projet}

\subsection{Utilisation des projets}

Gestionnaire de version

\section{Mode édition}

\subsection{Edition}

\subsection{Test}

Flux de donnée / présentation

\section{Les onglets}

\chapter{Les modèles}

\chapter{Les scriptes}

\chapter{Les extentions}

\section{Services}

\section{Generix}

\appendix
\chapter{Les raccourcis}

\chapter{Interface ECMAScript de XINX}

\end{document}
