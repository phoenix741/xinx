\documentclass[a4paper,12pt]{article}

%
% D-Bus Java Implementation
% Copyright (c) 2005-2006 Matthew Johnson
%
% This program is free software; you can redistribute it and/or modify it
% under the terms of either the GNU Lesser General Public License Version 2 or the
% Academic Free Licence Version 2.1.
%
% Full licence texts are included in the COPYING file with this program.
%

\usepackage{fullpage}
\usepackage{ifthen}

\ifx\pdfoutput\undefined
   \usepackage{hyperref}
\else
   \usepackage[
     bookmarks,
     bookmarksopen,
     bookmarksnumbered=false,
     colorlinks,
     pdfpagemode=None,
     urlcolor=black,
     citecolor=black,
     linkcolor=black,
     pagecolor=black,
     plainpages=false,
     pdfpagelabels,
   ]{hyperref}
\fi

\author{Matthew Johnson\\dbus@matthew.ath.cx}
\title{D-Bus programming in Java 1.5}

\begin{document}

\def\javadocroot/{http://dbus.freedesktop.org/doc/dbus-java/api/}

\maketitle

\tableofcontents
\listoffigures
\listoftables

\section{Introduction}

This document describes how to use the Java implementation of D-Bus. D-Bus is
an IPC mechanism which at a low level uses message passing over Unix Sockets or
IP. D-Bus models its messages as either function calls on remote objects, or
signals emitted from them.

Java is an object-oriented language and this implementation attempts to match
the D-Bus IPC model to the Java object model. The implementation also make
heavy use of threads and exceptions. The programmer should be careful to take
care of synchronisation issues in their code. All method calls by remote
programs on exported objects and all signal handlers are run in new threads.
Any calls on remote objects may throw {\tt DBusExecutionException}, which is a
runtime exception and so the compiler will not remind you to handle it.

The Java D-Bus API is also documented in the JavaDoc\footnote{\url{\javadocroot/}},
D-Bus is described in the
specification\footnote{http://dbus.freedesktop.org/doc/dbus-specification.html}
and the API documentation\footnote{http://dbus.freedesktop.org/doc/api/html/}.

\subsection{Protocol Implementation}

This library is a native Java implementation of the D-Bus protocol and not
a wrapper around the C reference implementation. 

\subsection{Dependencies}

This library requires Java 1.5-compatible VM and compiler (either Sun, or
ecj+jamvm with classpath-generics newer than 0.19) and the unix socket, debug
and hexdump libraries from \url{http://www.matthew.ath.cx/projects/java/}.

\subsection{D-Bus Terminology}

D-Bus has several notions which are exposed to the users of the Java
implementation.

\subsubsection{Bus Names}

Programs on the bus are issued a unique identifier by the bus. This
is guaranteed to be unique within one run of the bus, but is
assigned sequentially to each new connection. 

There are also so called well-known bus names which a device can
request on the bus. These are of the form {\em ``org.freedesktop.DBus''},
and any program can request them if they are not already owned.

\subsubsection{Interfaces}

All method calls and signals are specified using an interface,
similar to those in Java. When executing a method or sending a
signal you specify the interface the method belongs to. These are of
the form {\em ``org.freedesktop.DBus''}.

\subsubsection{Object Paths}

A program may expose more than one object which implements an
interface. Object paths of the form {\em ``/org/freedesktop/DBus''}
are used to distinguish objects.

\subsubsection{Member Names}

Methods and Signals have names which identify them within an
interface. D-Bus does not support method overloading, only one
method or signal should exist with each name.

\subsubsection{Errors}

A reply to any message may be an error condition. In which case you reply with
an error message which has a name of the form {\em
   ``org.freedesktop.DBus.Error.ServiceUnknown''}. 

\section{DBusConnection}

The {\tt
DBusConnection\footnote{\url{\javadocroot/org/freedesktop/dbus/DBusConnection.html}}}
class provides methods for connecting to the bus, exporting objects,
sending signals and getting references to remote objects.

{\tt DBusConnection} is a singleton class, multiple calls to {\tt
getConnection} will return the same bus connection. 

\begin{verbatim}
conn = DBusConnection.getConnection(DBusConnection.SESSION);
\end{verbatim}

This creates a connection to the session bus, or returns the
existing connection.

\begin{verbatim}
conn.addSigHandler(TestSignalInterface.TestSignal.class,
                   new SignalHandler());
\end{verbatim}

This sets up a signal handler for the given signal type.
SignalHandler.handle will be called in a new thread with an instance
of TestSignalInterface.TestSignal when that signal is recieved.

\begin{verbatim}
conn.sendSignal(new TestSignalInterface.TestSignal(
                     "/foo/bar/com/Wibble", 
                     "Bar", 
                     new UInt32(42)));
\end{verbatim}

This sends a signal of type {\tt TestSignalInterface.TestSignal},
from the object {\em ``/foo/bar/com/Wibble''} with the arguments {\em
``Bar''} and {\tt UInt32(42)}.

\begin{verbatim}
conn.exportObject("/Test", new testclass());
\end{verbatim}

This exports the {\tt testclass} object on the path {\em ``/Test''}

\begin{verbatim}
Introspectable intro = (Introspectable) conn.getRemoteObject(
                              "foo.bar.Test", "/Test",
                              Introspectable.class);
\end{verbatim}

This gets a reference to the {\em ``/Test''} object on the process with the
name {\em ``foo.bar.Test''} . The object implements the {\tt Introspectable}
interface, and calls may be made to methods in that interface as if it was a
local object.

\begin{verbatim}
String data = intro.Introspect();
\end{verbatim}

The Runtime Exception {\tt DBusExecutionException} may be thrown
by any remote method if any part of the execution fails.

\subsection{Asynchronous Calls}

Calling a method on a remote object is synchronous, that is the thread will
block until it has a reply. If you do not want to block you can use an
asynchronous call.

There are two ways of making asynchronous calls. You can either call the {\tt
callMethodAsync} function on the connection object, in which case you are
returned a {\tt
DBusAsyncReply\footnote{\url{\javadocroot/org/freedesktop/dbus/DBusAsyncReply.html}}}
object which can be used to check for a reply and get the return value.  This
is demonstrated in figure \ref{fig:async}.

\begin{figure}[htb]
\begin{center}
\begin{verbatim}
DBusAsyncReply<Boolean> stuffreply = 
   conn.callMethodAsync(remoteObject, "methodname", arg1, arg2);
...
if (stuffreply.hasReply()) {
   Boolean b = stuffreply.getReply();
   ...
}
\end{verbatim}
\end{center}
\caption{Calling an asynchronous method}
\label{fig:async}
\end{figure}

Alternatively, you can register a callback with the connection using the {\tt
callWithCallback} function on the connection object. In this case, your
callback class (implementing the {\tt
CallbackHandler\footnote{\url{\javadocroot/org/freedesktop/dbus/CallbackHandler.html}}}
interface will be called when the reply is returned from the bus.

\section{DBusInterface}

To call methods or expose methods on D-Bus you need to define them with their
exact signature in a Java interface. The full name of this interface must be
the same as the D-Bus interface they represent. In addition, D-Bus interface
names must contain at least one period. This means that DBusInterfaces cannot
be declared without being in a package. 

For example, if I want to expose methods on the interface {\em
``org.freedesktop.DBus''} I would define a Java interface in the package {\tt
org.freedesktop} called {\tt DBus}. This would be in the file {\tt
org/freedesktop/DBus.java} as normal. Any object wanting to export these
methods would implement {\tt org.freedesktop.DBus}.

Any interfaces which can be exported over D-Bus must extend {\tt
DBusInterface\footnote{\url{\javadocroot/org/freedesktop/dbus/DBusInterface.html}}}.
A class may implement more than one exportable interface, all public methods
declared in an interface which extend {\tt DBusInterface} will be exported.

A sample interface definition is given in figure~\ref{fig:interface}, and a
class which implements it in figure~\ref{fig:class}. More complicated
definitions can be seen in the test
classes\footnote{\url{\javadocroot/org/freedesktop/dbus/test/TestRemoteInterface2.java}
\url{\javadocroot/org/freedesktop/dbus/test/TestRemoteInterface.java}}.

All method calls by other programs on objects you export over D-Bus
are executed in their own thread.

{\tt DBusInterface} itself specifies one method \verb&boolean isRemote()&. If
this is executed on a remote object it will always return true. Local objects
implementing a remote interface must implement this method to return false.

\begin{figure}[htb]
\begin{center}
\begin{verbatim}
package org.freedesktop;
import org.freedesktop.dbus.UInt32;
import org.freedesktop.dbus.DBusInterface;

public interface DBus extends DBusInterface
{
   public boolean NameHasOwner(String name);
   public UInt32 RequestName(String name, UInt32 flags);
}
\end{verbatim}
\end{center}
\caption{An interface which exposes two methods}
\label{fig:interface}
\end{figure}

\begin{figure}[htb]
\begin{center}
\begin{verbatim}
package my.real.implementation;
import org.freedesktop.dbus.DBus;
import org.freedesktop.dbus.UInt32;

public class DBusImpl implements DBus
{
   Vector<String> names;
   public boolean NameHasOwner(String name)
   {
      if (names.contains(name)) return true;
      else return false;
   }
   public UInt32 RequestName(String name, UInt32 flags)
   {
      names.add(name);
      return new UInt32(0);
   }
   public boolean isRemote() { return false; }
}
\end{verbatim}
\end{center}
\caption{A class providing a real implementation which can be exported}
\label{fig:class}
\end{figure}

\subsection{Interface name overriding}

It is highly recommended that the Java interface and package name match the
D-Bus interface. However, if, for some reason, this is not possible then the
name can be overridden by use of an Annotation.

To override the Java interface name you should add an annotation to the
interface of {\tt
DBusInterfaceName\footnote{\url{\javadocroot/org/freedesktop/dbus/DBusInterfaceName.html}}}
with a value of the desired D-Bus interface name. An example of this can be seen in
figure \ref{fig:interfacename}.

\begin{figure}[htb]
\begin{center}
\begin{verbatim}
package my.package;
import org.freedesktop.dbus.DBusInterface;
import org.freedesktop.dbus.DBusInterfaceName;

@DBusInterfaceName("my.otherpackage.Remote")
public interface Remote extends DBusInterface
{
   ...
}
\end{verbatim}
\end{center}
\caption{Overloading the name of an interface.}
\label{fig:interfacename}
\end{figure}

\section{DBusSignal}

Signals are also declared as part of an interface. The Java API
models these as inner classes within an interface. The containing
interface must extend {\tt DBusInterface}, and the inner classes
representing the signals must extend {\tt
DBusSignal\footnote{\url{\javadocroot/org/freedesktop/dbus/DBusSignal.html}}}.
The Signal name is derived from the name of this inner class, and
the interface from its containing interface.

Signals can take parameters as methods can (although they cannot return
      anything). For the reflection to work, a Signal declare a single
constructor of the correct type. The constructor must  take the object path
they are being emitted from as their first (String) argument, followed by the
other parameters in order. They must also call the superclass constructor with
the same parameters.  A full definition of a signal can be seen in
figure~\ref{fig:signal}. Again, more complicated definitions are available in
the test
classes\footnote{\url{\javadocroot/org/freedesktop/dbus/test/TestSignalInterface.html}}.

\begin{figure}[htb]
\begin{center}
\begin{verbatim}
package org.freedesktop;
import org.freedesktop.dbus.DBusInterface;
import org.freedesktop.dbus.DBusSignal;
import org.freedesktop.dbus.exceptions.DBusException;

public interface DBus extends DBusInterface
{
   public class NameAcquired extends DBusSignal
   {
      public final String name;
      public NameAcquired(String path, String name) 
                              throws DBusException
      {
         super(path, name);
         this.name = name;
      }
   }
}
\end{verbatim}
\end{center}
\caption{A Signal with one parameter}
\label{fig:signal}
\end{figure}

\section{DBusExecutionException}

If you wish to report an error condition in a method call you can throw an
instance of {\tt
DBusExecutionException\footnote{\url{\javadocroot/org/freedesktop/dbus/DBusExecutionException.html}}}.
This will be sent back to the caller as an error message, and the error name is
taken from the class name of the exception. For example, if you wanted to
report an unknown method you would define an exception as in figure
\ref{fig:exceptiondef} and then throw it in your method as in figure
\ref{fig:exceptioncall}.

\begin{figure}[htb]
\begin{center}
\begin{verbatim}
package org.freedesktop.DBus.Error;
import org.freedesktop.dbus.exceptions.DBusExecutionException;

public class UnknownMethod extends DBusExecutionException
{
   public UnknownMethod(String message)
   {
      super(message);
   }
}
\end{verbatim}
\end{center}
\caption{An Exception}
\label{fig:exceptiondef}
\end{figure}

\begin{figure}[htb]
\begin{center}
\begin{verbatim}
...
public void throwme() throws org.freedesktop.DBus.Error.UnknownMethod
{
   throw new org.freedesktop.DBus.Error.UnknownMethod("hi");
}
...
\end{verbatim}
\end{center}
\caption{Throwing An Exception}
\label{fig:exceptioncall}
\end{figure}

If you are calling a remote method and you want to handle such an error you can
simply catch the exception as in figure \ref{fig:exceptioncatch}. 

\begin{figure}[htb]
\begin{center}
\begin{verbatim}
...
try {
   remote.throwme();
} catch (org.freedesktop.DBus.Error.UnknownMethod UM) {
   ...
}
...
\end{verbatim}
\end{center}
\caption{Catching An Exception}
\label{fig:exceptioncatch}
\end{figure}

\section{DBusSigHandler}

To handle incoming signals from other programs on the Bus you must register a
signal handler. This must implement {\tt
DBusSigHandler\footnote{\url{\javadocroot/org/freedesktop/dbus/DBusSigHandler.html}}}
and provide an implementation for the handle method. An example Signal Handler
is in figure~\ref{fig:handler}. Signal handlers should be parameterised with
the signal they are handling. If you want a signal handler to handle multiple
signals you can leave out the parameterisation and use {\tt instanceof} to
check the type of signal you are handling. Signal handlers will be run in their
own thread.

\begin{figure}[htb]
\begin{center}
\begin{verbatim}
import org.freedesktop.dbus.DBusSignal;
import org.freedesktop.dbus.DBusSigHandler;

public class Handler extends DBusSigHandler<DBus.NameAcquired>
{
   public void handle(DBus.NameAcquired sig)
   {
         ...
   }
}
\end{verbatim}
\end{center}
\caption{A Signal Handler}
\label{fig:handler}
\end{figure}


\section{D-Bus Types}

D-Bus supports a number of types in its messages, some which Java
supports natively, and some which it doesn't. This library provides
a way of modelling the extra D-Bus Types in Java. The full list of
types and what D-Bus type they map to is in table \ref{table:types}.

\subsection{Basic Types}

All of Java's basic types are supported as parameters and return types to methods, and as parameters to signals. These can be used in either their primitive or wrapper types.

\subsubsection{Unsigned Types}

D-Bus, like C and similar languages, has a notion of unsigned numeric
types. The library supplies {\tt
UInt16\footnote{\url{\javadocroot/org/freedesktop/dbus/UInt16.html}}}, {\tt
UInt32} and {\tt UInt64} classes to represent these new basic types.

\subsection{Strings}

D-Bus also supports sending Strings. When mentioned below, Strings
count as a basic type.

\subsubsection{String Comparisons}

There may be some problems with comparing strings received over D-Bus with
strings generated locally when using the String.equals method. This is due to
how the Strings are generated from a UTF8 encoding. The recommended way to
compare strings which have been sent over D-Bus is with the {\tt
java.text.Collator} class. Figure \ref{fig:collator} demonstrates its use.

\begin{figure}[htb]
\begin{center}
\begin{verbatim}
String rname = remote.getName();
Collator col = Collator.getInstance();
col.setDecomposition(Collator.FULL_DECOMPOSITION);
col.setStrength(Collator.PRIMARY);
if (0 != col.compare("Name", rname))
   fail("getName return value incorrect");
\end{verbatim}
\end{center}
\caption{Comparing strings with {\tt java.text.Collator}.}
\label{fig:collator}
\end{figure}

\subsection{Arrays}

You can send arrays of any valid D-Bus Type over D-Bus. These can either be
declared in Java as arrays (e.g. \verb&Integer[]& or \verb&int[]&) or as Lists
(e.g. \verb&List<String>&). All lists {\bf must} be parameterised with their
type in the source (reflection on this is used by the library to determine
their type). Also note that arrays cannot be used as part of more complex type,
only Lists (for example \verb&List<List<String>>&).

\subsection{Maps}

D-Bus supports a dictionary type analogous to the Java Map type. This
has the additional restriction that only basic types can be used as
the key (including String). Any valid D-Bus type can be the value. As
with lists, maps must be fully parameterised. (e.g.
\verb&Map<Integer, String>&).

\subsection{Variants}

D-Bus has support for a Variant type. This is similar to declaring that a
method takes a parameter of type {\tt Object}, in that a Variant may contain
any other type. Variants can either be declared using the {\tt
Variant\footnote{\url{\javadocroot/org/freedesktop/dbus/Variant.html}}} class, or as
a Type Variable. In the latter case the value is automatically unwrapped and
passed to the function. Variants in compound types (Arrays, Maps, etc) must be
declared using the Variant class with the full type passed to the Variant
constructor and manually unwrapped.

Both these methods use variants:

\begin{verbatim}
public void display(Variant v);
public <T> int hash(T v);
\end{verbatim}

\subsection{Structs}

D-Bus has a struct type, which is a collection of other types. Java
does not have an analogue of this other than fields in classes, and
due to the limitation of Java reflection this is not sufficient. The
library declares a {\tt
Struct\footnote{\url{\javadocroot/org/freedesktop/dbus/Struct.html}}} class which can be used to create structs.
To define a struct you extend the {\tt Struct} class and define fields for each member of the struct.
These fields then need to be annotated in the order which they appear in the struct (class fields do not have a defined order). You must also define a single constructor which takes the contents of he struct in order. This is best demonstrated by an example.
Figure~\ref{fig:struct} shows a Struct definition, and
figure~\ref{fig:structmethod} shows this being used as a parameter
to a method.

\begin{figure}[htb]
\begin{center}
\begin{verbatim}
package org.freedesktop.dbus.test;

import org.freedesktop.dbus.DBusException;
import org.freedesktop.dbus.Position;
import org.freedesktop.dbus.Struct;

public final class TestStruct extends Struct
{
   @Position(0)
   public final String a;
   @Position(1)
   public final int b;
   @Position(2)
   public final String c;
   public Struct3(String a, int b, String c)
   {
      this.a = a;
      this.b = b;
      this.c = c;
   }
}
\end{verbatim}
\end{center}
\caption{A Struct with three elements}
\label{fig:struct}
\end{figure}

\begin{figure}[htb]
\begin{center}
\begin{verbatim}
public void do(TestStruct data);
\end{verbatim}
\end{center}
\caption{A struct as a parameter to a method}
\label{fig:structmethod}
\end{figure}


Section~\ref{sec:create} describes how these can be automatically
generated from D-Bus introspection data.

\subsection{Objects}

You can pass references to exportable objects round using their object paths.
To do this in Java you declare a type of {\tt DBusInterface}. When the library
receive- an object path it will automatically convert it into the object you
are exporting with that object path. You can pass remote objects back to their
processes in a similar fashion.

Using a parameter of type {\tt DBusInterface} can cause the automatic creation
of a proxy object using introspection. If the remote app does not support
introspection, or the object does not exist at the time you receive the message
then this will fail. In that case the parameter can be declared to be of type
{\tt Path}. In this case no automatic creation will be performed and you can
get the path as a string with either the {\tt getPath} or {\tt toString} methods
on the {\tt Path} object.


\subsection{Multiple Return Values}

D-Bus also allows functions to return multiple values, a concept not supported
by Java. This has been solved in a fashion similar to the struct, using a {\tt
Tuple\footnote{\url{\javadocroot/org/freedesktop/dbus/Tuple.html}}}
class. Tuples are defined as generic tuples which can be parameterised for
different types and just need to be defined of the appropriate length. This can
be seen in figure~\ref{fig:tuple} and a call in figure~\ref{fig:tuplemethod}.
Again, these can be automatically generated from introspection data.

\begin{figure}[htb]
\begin{center}
\begin{verbatim}
import org.freedesktop.dbus.Tuple;

public final class TestTuple<A, B, C> extends Tuple
{
   public final A a;
   public final B b;
   public final C c;
   public TestTuple(A a, B b, C c)
   {
      this.a = a;
      this.b = b;
      this.c = c;
   }
}
\end{verbatim}
\end{center}
\caption{A 3-tuple}
\label{fig:tuple}
\end{figure}

\begin{figure}[htb]
\begin{center}
\begin{verbatim}
public ThreeTuple<String, Integer, Boolean> status(int item);
\end{verbatim}
\end{center}
\caption{A Tuple being returned from a method}
\label{fig:tuplemethod}
\end{figure}

\subsection{Full list of types}

Table \ref{table:types} contains a full list of all the Java types and their corresponding D-Bus types.

\begin{table}
\begin{center}
\begin{tabular}{l|l}
\bf Java Type & \bf D-Bus Type \\
\hline
Byte	&	DBUS\_TYPE\_BYTE	\\
byte	&	DBUS\_TYPE\_BYTE	\\
Boolean	&	DBUS\_TYPE\_BOOLEAN	\\
boolean	&	DBUS\_TYPE\_BOOLEAN	\\
Short	&	DBUS\_TYPE\_INT16	\\
short	&	DBUS\_TYPE\_INT16	\\
UInt16	&	DBUS\_TYPE\_UINT16	\\
int	&	DBUS\_TYPE\_INT32	\\
Integer	&	DBUS\_TYPE\_INT32	\\
UInt32	&	DBUS\_TYPE\_UINT32	\\
long	&	DBUS\_TYPE\_INT64	\\
Long	&	DBUS\_TYPE\_INT64	\\
UInt64	&	DBUS\_TYPE\_UINT64	\\
double	&	DBUS\_TYPE\_DOUBLE	\\
Double	&	DBUS\_TYPE\_DOUBLE	\\
String	&	DBUS\_TYPE\_STRING	\\
Path	&	DBUS\_TYPE\_OBJECT\_PATH	\\
$<$T$>$	&	DBUS\_TYPE\_VARIANT	\\
Variant	&	DBUS\_TYPE\_VARIANT	\\
? extends Struct	&	DBUS\_TYPE\_STRUCT	\\
?$[$~$]$	&	DBUS\_TYPE\_ARRAY	\\
? extends List	&	DBUS\_TYPE\_ARRAY	\\
? extends Map	&	DBUS\_TYPE\_DICT	\\
? extends DBusInterface	&	DBUS\_TYPE\_OBJECT\_PATH	\\
Type$[$~$]$	&	DBUS\_TYPE\_SIGNATURE	\\
\end{tabular}
\end{center}
\caption{Mapping between Java types and D-Bus types}
\label{table:types}
\end{table}

\subsubsection{float}

Currently the D-Bus reference implementation does not support a native
single-precision floating point type. Along with the C\# implementation of the
protocol, the Java implementation supports this extension to the protocol. By
default, however, the library operates in compatibility mode and converts all
floats to the double type. To disable compatibility mode export the environment
variable {\tt DBUS\_JAVA\_FLOATS=true}.

\section{Annotations}

You can annotate your D-Bus methods as in figure \ref{fig:annotation} to provide hints to other users of your API. Common annotations are listed in table \ref{tab:annotations}.

\begin{figure}[htb]
\begin{center}
\begin{verbatim}
package org.freedesktop;
import org.freedesktop.dbus.UInt32;
import org.freedesktop.dbus.DBusInterface;

@org.freedesktop.DBus.Description("Some Methods");
public interface DBus extends DBusInterface
{
   @org.freedesktop.DBus.Description("Check if the name has an owner")
   public boolean NameHasOwner(String name);
   @org.freedesktop.DBus.Description("Request a name")
   @org.freedesktop.DBus.Deprecated()
   public UInt32 RequestName(String name, UInt32 flags);
}
\end{verbatim}
\end{center}
\caption{An annotated method}
\label{fig:annotation}
\end{figure}


\begin{table}[htb]
\begin{tabular}{l|l}
{\bf Name} & {\bf Meaning} \\
\hline
org.freedesktop.DBus.Description & Provide a short 1-line description \\
      & of the method or interface. \\
org.freedesktop.DBus.Deprecated & This method or interface is Deprecated. \\
org.freedesktop.DBus.Method.NoReply & This method may be called and returned \\
   & without waiting for a reply. \\
org.freedesktop.DBus.Method.Error & This method may throw the listed Exception\\
   & in addition to the standard ones. \\
\end{tabular}
\caption{Common Annotations}
\label{tab:annotations}
\end{table}

\section{DBusSerializable}

Some people may want to be able to pass their own objects over D-Bus. Obviously
only raw D-Bus types can be sent on the message bus itself, but the Java
implementation allows the creation of serializable objects which can be passed to
D-Bus functions and will be converted to/from D-Bus types by the library.

To create such a class you must implement the {\tt
DBusSerializable\footnote{\url{\javadocroot/org/freedesktop/dbus/DBusSerializable.html}}}
class and provide two methods and a zero-argument constructor. The first method
has the signature {\tt public Object\[\] serialize() throws DBusException} and
the second must be called {\tt deserialize}, return {\tt null} and take as it's
arguments exactly all the dbus types that are being serialized to {\em in
order} and {\em with parameterization}. The serialize method should return the
class properties you wish to serialize, correctly formatted for the wire ({\tt
DBusConnection.convertParameters()} can help with this), in order in an Object
array.

An example of a serializable class can be seen in figure~\ref{fig:serializable}.


\begin{figure}[htb]
\begin{center}
\begin{verbatim}
import java.lang.reflect.Type;

import java.util.List;
import java.util.Vector;

import org.freedesktop.dbus.DBusConnection;
import org.freedesktop.dbus.DBusSerializable;

public class TestSerializable implements DBusSerializable
{
   private int a;
   private String b;
   private Vector<Integer> c;
   public TestSerializable(int a, String b, Vector<Integer> c)
   {
      this.a = a;
      this.b = b.toString();
      this.c = c;
   }
   public TestSerializable() {}
   public void deserialize(int a, String b, List<Integer> c)
   {
      this.a = a;
      this.b = b;
      this.c = new Vector<Integer>(c);
   }
   public Object[] serialize() 
   {
      return new Object[] { a, b, c };
   }
   public int getInt() { return a; }
   public String getString() { return b; }
   public Vector<Integer> getVector() { return c; }
   public String toString()
   {
      return "TestSerializable{"+a+","+b+","+c+"}";
   }
}
\end{verbatim}
\end{center}
\caption{A serializable class}
\label{fig:serializable}
\end{figure}



\section{CreateInterface}
\label{sec:create}

D-Bus provides a method to get introspection data on a remote object, which
describes the interfaces, methods and signals it provides.  This introspection
data is in XML
format\footnote{http://dbus.freedesktop.org/doc/dbus-specification.html\#introspection-format}.
The library automatically provides XML introspection data on all objects which
are exported by it.  Introspection data can be used to create Java interface
definitions automatically.

The {\tt
CreateInterface\footnote{\url{\javadocroot/org/freedesktop/dbus/CreateInterface.html}}}
class will automatically create Java source files from an XML file
containing the introspection data, or by querying the remote object
over D-Bus.  CreateInterface can be called from Java code, or can be run as a
stand alone program.  

The syntax for the CreateInterface program is

\begin{verbatim}
CreateInterface [--system] [--session] [--create-files] 
                  <bus name> <object>
CreateInterface [--create-files] <introspection-file.xml>
\end{verbatim}

The Java source code interfaces will be written to the standard ouput. If the
{\tt --create-files} option is specified the correct files in the
correct directory structure will be created.

\subsection{Nested Interfaces}

In some cases there are nested interfaces. In this case CreateInterface will
not correctly create the Java equivalent. This is because Java cannot have both
a class and a package with the same name. The solution to this is to create
nested classes in the same file.

An example would be the Hal interface:

\begin{verbatim}
<interface name="org.freedesktop.Hal.Device">
   ...
</interface>
<interface name="org.freedesktop.Hal.Device.Volume">
   ...
</interface>
\end{verbatim}

When converted to Java you would just have one file {\tt
org/freedesktop/Hal/Device.java} in the package {\tt org.freedesktop.Hal},
which would contain one class and one nested class:

\begin{verbatim}
public interface Device extends DBusInterface {
   public interface Volume extends DBusInterface {
      ... methods in Volume ...
   }
   ... methods in Device ...
}
\end{verbatim}

\section{Debugging}

It is possible to enable debugging in the library. This will be a lot slower,
but can print a lot of useful information for debugging your program.

To enable a debug build compile with {\tt DEBUG=enable}. This will then need to be
enabled at runtime by using the debug jar with debugging enabled (usually
installed as debug-enable.jar alongside the normal jar).

Running a program which uses this library will print some informative messages.
More verbose debug information can be got by supplying a custom debug
configuration file. This should be placed in the file {\tt debug.conf} and has the
format:

\begin{verbatim}
classname = LEVEL
\end{verbatim}

Where {\tt classname} is either the special word {\tt ALL} or a full class name
like {\tt org.freedesktop.dbus} and {\tt LEVEL} is one of {\tt NONE}, {\tt
CRIT}, {\tt ERR}, {\tt WARN}, {\tt INFO}, {\tt DEBUG}, {\tt VERBOSE}, {\tt
YES}, {\tt ALL} or {\tt TRUE}. This will set the debug level for a particular
class. Any messages from that class at that level or higher will be printed.
Verbose debugging is extremely verbose.

In addition, setting the environment variable {\tt
DBUS\_JAVA\_EXCEPTION\_DEBUG} will cause all exceptions which are handled
internally to have their stack trace printed when they are handled. This will
happen unless debugging has been disabled for that class.

\section{Peer to Peer}

It is possible to connect two applications together directly without using a
bus. D-Bus calls this a peer-to-peer connection.

The Java implementation provides an alternative to the {\tt DBusConnection}
class, the {\tt
DirectConnection\footnote{\url{\javadocroot/org/freedesktop/dbus/DirectConnection.html}}}
class. This allows you to connect two applications together directly without
the need of a bus.

When using a DirectConnection rather than a DBusConnection most operations are
the same. The only things which differ are how you connect and the operations
which depend on a bus. Since peer connections are only one-to-one there is no
destination or source address to messages. There is also no {\tt
org.freedesktop.DBus} service running on the bus.

\subsection{Connecting to another application}

To connect with a peer connection one of the two applications must be listening
on the socket and the other connecting. Both of these use the same method to
instantiate the {\tt DirectConnection} but with different addresses. To listen
rather than connect you add the {\em ``listen=true''} parameter to the address.
Listening and connecting can be seen in figures \ref{fig:p2plisten} and
\ref{fig:p2pconnect} respectively. Listening will block at creating the
connection until the other application has connected.

{\tt DirectConnection} also provides a {\tt createDynamicSession} method which
generates a random abstract unix socket address to use.

\begin{figure}[htb]
\begin{center}
\begin{verbatim}
DirectConnection dc = new DirectConnection("unix:path=/tmp/dbus-ABCXYZ,listen=true");
\end{verbatim}
\end{center}
\caption{Listening for a peer connection}
\label{fig:p2plisten}
\end{figure}

\begin{figure}[htb]
\begin{center}
\begin{verbatim}
DirectConnection dc = new DirectConnection("unix:path=/tmp/dbus-ABCXYZ");
\end{verbatim}
\end{center}
\caption{Connecting to a peer connection}
\label{fig:p2pconnect}
\end{figure}

\subsection{Getting Remote Objects}

Getting a remote object is essentially the same as with a bus connection,
except that you do not have to specify a bus name, only an object path. There
is also no {\tt getPeerRemoteObject} method, since there can only be one peer
on this connection.

\begin{figure}[htb]
\begin{center}
\begin{verbatim}
RemoteInterface remote = dc.getRemoteObject("/Path");
remote.doStuff();
\end{verbatim}
\end{center}
\caption{Getting a remote object on a peer connection}
\label{fig:p2premote}
\end{figure}

The rest of the API is the same for peer connections as bus connections,
ommiting any bus addresses.

\section{Low-level API}

In very rare circumstances it may be neccessary to deal directly with messages
on the bus, rather than with objects and method calls. This implementation
gives the programmer access to this low-level API but its use is strongly
recommended against.

To use the low-level API you use a different set of classes than with the
normal API.

\subsection{Transport}

The {\tt Transport\footnote{\url{\javadocroot/org/freedesktop/dbus/Transport.html}}}
class is used to connect to the underlying transport with a bus address and to
send and receive messages. 

You connect by either creating a {\tt Transport} object with the bus address as
the parameter, or by calling {\tt connect} with the address later. Addresses
are represented using the {\tt BusAddress} class.

Messages can be read by calling {\tt transport.min.readMessage()} and written
by using the {\tt transport.mout.writeMessage(m)} methods.

\subsection{Message}

{\tt Message\footnote{\url{\javadocroot/org/freedesktop/dbus/Message.html}}} is the
superclass of all the classes representing a message. To send a message you
need to create a subclass of this object. Possible message types are: {\tt
MethodCall}, {\tt MethodReturn}, {\tt Error} and {\tt DBusSignal}. Constructors
for these vary, but they are basically similar to the {\tt MethodCall} class.

All the constructors have variadic parameter lists with the last of the
parameters being the signature of the message body and the parameters which
make up the body. If the message has an empty body then the last parameter must
be null. Reading and writing messages is not thread safe.

Messages can be read either in blocking or non-blocking mode. When reading a
message in non-blocking mode, if a full message has not yet been read from the
transport the method will return null. Messages are instantiated as the correct
message type, so {\tt instanceof} will work on the returned object. Blocking
mode can be enabled with an extra parameter to the Transport constructor.

Figure \ref{fig:lowlevel} shows how to connect to a bus, send the (required)
initial `Hello' message and 
call a method with two parameters.

\begin{figure}[htb]
\begin{center}
\begin{verbatim}
BusAddress address = new BusAddress(
         System.getenv("DBUS_SESSION_BUS_ADDRESS"));
Transport conn = new Transport(address, true);

Message m = new MethodCall("org.freedesktop.DBus", "/org/freedesktop/DBus", 
                           "org.freedesktop. DBus", "Hello", (byte) 0, null);
conn.mout.writeMessage(m);

m = conn.min.readMessage();
System.out.println("Response to Hello is: "+m);

m = new MethodCall("org.freedesktop.DBus", "/org/freedesktop/DBus", 
                   "org.freedesktop.DBus", "RequestName", (byte) 0, 
                   "su", "org.testname", 0);
conn.mout.writeMessage(m);

conn.disconnect();
\end{verbatim}
\end{center}
\caption{Low-level usage}
\label{fig:lowlevel}
\end{figure}

\section{Examples}

As an example here are a complete set of interfaces for the
bluemon\footnote{http://www.matthew.ath.cx/projects/bluemon} daemon,
which communicates over D-Bus. These interfaces were all created by
querying introspection data over the bus.

\newpage

\begin{figure}[!h]
\begin{center}
\begin{verbatim}
package cx.ath.matthew.bluemon;
import org.freedesktop.dbus.DBusInterface;
import org.freedesktop.dbus.UInt32;
public interface Bluemon extends DBusInterface
{
  public Triplet<String, Boolean, UInt32> 
  Status(String address);
}
\end{verbatim}
\end{center}
\caption{cx/ath/matthew/bluemon/Bluemon.java}
\end{figure}

\newpage

\begin{figure}[!h]
\begin{center}
\begin{verbatim}
package cx.ath.matthew.bluemon;
import org.freedesktop.dbus.DBusInterface;
import org.freedesktop.dbus.DBusSignal;
import org.freedesktop.dbus.exceptions.DBusException;
public interface ProximitySignal extends DBusInterface
{
   public static class Connect extends DBusSignal
   {
      public final String address;
      public Connect(String path, String address) 
                                 throws DBusException
      {
         super(path, address);
         this.address = address;
      }
   }
   public static class Disconnect extends DBusSignal
   {
      public final String address;
      public Disconnect(String path, String address)
                                 throws DBusException
      {
         super(path, address);
         this.address = address;
      }
   }
}
\end{verbatim}
\end{center}
\caption{cx/ath/matthew/bluemon/ProximitySignal.java}
\end{figure}

\newpage

\begin{figure}[!h]
\begin{center}
\begin{verbatim}
package cx.ath.matthew.bluemon;
import org.freedesktop.dbus.Tuple;
/** Just a typed container class */
public final class Triplet <A,B,C> extends Tuple
{
   public final A a;
   public final B b;
   public final C c;
   public Triplet(A a, B b, C c)
   {
      super(a, b, c);
      this.a = a;
      this.b = b;
      this.c = c;
   }
}
\end{verbatim}
\end{center}
\caption{cx/ath/matthew/bluemon/Triplet.java}
\end{figure}

\newpage

\begin{figure}[!h]
\begin{center}
\begin{verbatim}
package cx.ath.matthew.bluemon;
import org.freedesktop.dbus.DBusConnection;
import org.freedesktop.dbus.DBusSigHandler;
import org.freedesktop.dbus.DBusSignal;
import org.freedesktop.dbus.UInt32;
import org.freedesktop.dbus.exceptions.DBusException;

public class Query {
   public static void main(String[] args) {
      String btid;
      Triplet<String, Boolean, UInt32> rv = null;
      
      if (0 == args.length) btid = "";
      else btid = args[0];
      
      DBusConnection conn = null;
      try {
         conn = DBusConnection.getConnection(DBusConnection.SYSTEM);
      } catch (DBusException De) {
         System.exit(1);
      }     
      Bluemon b = (Bluemon) conn.getRemoteObject(
            "cx.ath.matthew.bluemon.server", 
            "/cx/ath/matthew/bluemon/Bluemon", Bluemon.class);
      try {
         rv = b.Status(btid);
      } catch (RuntimeException Re) {
         System.exit(1);
      }
      String address = rv.a;
      boolean status = rv.b;
      int level = rv.c.intValue();

      if (status)
         System.out.println("Device "+address+
                            " connected with level "+level);
      else
         System.out.println("Device "+address+" not connected");
      conn.disconnect();
   }
}
\end{verbatim}
\end{center}
\caption{cx/ath/matthew/bluemon/Query.java}
\end{figure}

\newpage

\begin{figure}[!h]
\begin{center}
\begin{verbatim}
/* cx/ath/matthew/bluemon/Client.java */
package cx.ath.matthew.bluemon;

import org.freedesktop.dbus.DBusConnection;
import org.freedesktop.dbus.DBusSigHandler;
import org.freedesktop.dbus.DBusSignal;
import org.freedesktop.dbus.exceptions.DBusException;

public class Client implements DBusSigHandler
{
   public void handle(DBusSignal s)
   {
      if (s instanceof ProximitySignal.Connect)
         System.out.println("Got a connect for "
               +((ProximitySignal.Connect) s).address);
      else if (s instanceof ProximitySignal.Disconnect)
         System.out.println("Got a disconnect for "
               +((ProximitySignal.Disconnect) s).address);
   }
   public static void main(String[] args) 
   {
      System.out.println("Creating Connection");
      DBusConnection conn = null;
      try {
         conn = DBusConnection
                  .getConnection(DBusConnection.SYSTEM);
      } catch (DBusException DBe) {
         System.out.println("Could not connect to bus");
         System.exit(1);
      }
      
      try {
         conn.addSigHandler(ProximitySignal.Connect.class, 
         new Client());
         conn.addSigHandler(ProximitySignal.Disconnect.class, 
         new Client());
      } catch (DBusException DBe) {
         conn.disconnect();
         System.exit(1);
      }
   }
}
\end{verbatim}
\end{center}
\caption{cx/ath/matthew/bluemon/Client.java}
\end{figure}

\newpage

\section{Credits}

This document and the Java API were written by and are copyright to
Matthew Johnson. Much help and advice was provided by the members of
the \#dbus irc channel. Comments, corrections and patches can be sent
to dbus-java@matthew.ath.cx.

\end{document}


